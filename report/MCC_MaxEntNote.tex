\documentclass[11pt]{article}   
\usepackage{geometry} 
\geometry{letterpaper}  
%\usepackage[parfill]{parskip}    		% Activate to begin paragraphs with an empty line rather than an indent

\usepackage{graphicx}
\usepackage{subfigure}
\usepackage{mathtools}

\usepackage{amsmath,amssymb}


% Theorems etc.
%\newtheorem{thm}{Theorem}
%\newtheorem{lemma}[theorem]{Lemma}
%\newtheorem{corollary}{Corollary}[theorem]
\newtheorem{remark}{Remark}%[theorem]

% mathtools-based definitions
\DeclarePairedDelimiter\ceil{\lceil}{\rceil}
\DeclarePairedDelimiter\floor{\lfloor}{\rfloor}
\DeclarePairedDelimiter\cardinality{\lvert}{\rvert}

% miscellaneous macros for probability
\DeclareMathOperator{\var}{Var}
\DeclareMathOperator{\stdev}{SD}
\DeclareMathOperator{\mean}{E}
\DeclareMathOperator{\prb}{Pr}
\DeclareMathOperator*{\maximize}{maximize}
\DeclareMathOperator*{\minimize}{minimize}
\newcommand{\prob}[1]{\prb[#1]}
\newcommand{\xC}{\mathcal{C}}
\newcommand{\xD}{\mathcal{D}}
\newcommand{\xB}{\mathcal{B}}
\newcommand{\xM}{\mathcal{M}}
\newcommand{\xP}{\mathcal{P}}
\newcommand{\xR}{\mathcal{R}}
\newcommand{\xS}{\mathcal{S}}
\newcommand{\muhat}{\hat{\mu}}
\newcommand{\phat}{\hat{p}}
\newcommand{\yI}{\tilde{I}}
\newcommand{\yL}{\tilde{L}}
\newcommand{\yW}{\tilde{W}}
\let\eps\epsilon
\newcommand{\darg}{\,\cdot\,}

% Space and list control
\frenchspacing %remove extra space after periods (annoying and out of style)
%\makeatletter
%\renewcommand{\@listI}{%
%\leftmargin=25pt
%\rightmargin=0pt
%\labelsep=5pt
%\labelwidth=20pt
%\itemindent=0pt
%\listparindent=0pt
%\topsep=2pt plus 2pt minus 4pt
%\partopsep=0pt plus 1pt minus 1pt
%\parsep=2pt plus 1pt
%\itemsep=\parsep}
%\makeatother
\newenvironment{romlist}{
  \def\labelenumi{(\theenumi)}
  \def\theenumi{\roman{enumi}}
  \def\labelenumii{(\theenumii)}
  \def\theenumii{\roman{enumii}}
  \begin{enumerate}
  }{\end{enumerate}}

\newcommand{\BigCrunch}{\vspace*{-1.5em}}
\newcommand{\Crunch}{\vspace*{-1em}}
\newcommand{\SmallCrunch}{\vspace*{-1ex}}

\title{Statistical estimation methodologies for patients with multiple chronic conditions}
\author{Peter J. Haas, Ninad Khargonkar, Hari Balasubramanian, Chaitra Gopalappa}
%\date{}

\begin{document}
\maketitle


\section {Motivation}

More than half of the US population has at least one chronic condition and over 30\% have two or more chronic conditions. Patients with 2 more chronic conditions account for 70\% of the US healthcare expenditures. Medical interventions and policy initiatives generally focus on a single health condition rather than combinations of health conditions. This approach risks ignoring unique challenges faced by patients with multiple conditions. The Centers for Disease Control and Prevention notes that "as the number of chronic conditions in an individual increases, the risks of the following outcomes also increase: mortality, poor functional status, unnecessary hospitalizations, adverse drug events, duplicative tests, and conflicting medical advice." 

While the chronic conditions that are individually most prevalent in the US population are well known, their \emph {joint prevalence}, that is the probability estimate of how frequently a particular combination of conditions co-occurs in the population is harder to estimate. This is because the number of individuals in a data sample who have a specific combination of conditions tends to be low. For instance, in the US, it is estimated that  only 5.0\% of the population has 4 chronic conditions. This means that in a dataset of 1000 nationally representative individuals only 50 individuals (approximately) will have 4 chronic conditions. If we look at a specific combination of 4 chronic conditions (say diabetes, hypertension, cholesterol and arthritis), the number of individuals who have these four conditions will be even smaller. Indeed, a vast majority of combinations of four will have no representation at all. Estimating joint prevalence directly through data therefore risks underestimating the true prevalence and thereby the impact that such patients have on the health system.      

The above discussion is one example of a statistical estimation challenge related to patients with multiple chronic conditions (MCC). Other open questions include identifying combinations of chronic conditions that occur most frequently in a population, workload estimation for specialty providers seen by MCC patients, and evaluation of new delivery models that coordinate care more effectively and improve outcomes. These research questions are summarized below:   

\begin{enumerate}
\item Which combination of $k$ chronic conditions is the most prevalent in a dataset? [Market Basket Analysis -- we have some preliminary results]
\item How do we accurately estimate the joint prevalence probabilities for a combination of $k$ conditions? Since chronic conditions are highly age-dependent, how do we estimate transition probabilities from one year to the next? [Max Ent and Markov transitions with Max Ent -- we will have some preliminary results]
\item How do combinations of chronic conditions differ with regard to capacity and resources needed from different speciality providers? [Yet to be figured out.]
\item How do we measure the impact of interventions that target a combination of conditions rather than individual conditions? [Yet to be figured out.]  
\end{enumerate}
 

\section{MaxEnt Preliminaries}

Consider a disease model in which an individual can simultaneously have up to $k> 1$ diseases. We represent the state of an individual by a vector $r=(r_1,\ldots,r_k)$, where, for $j\in[1..k]$, we set $r_j=1$ if the individual has disease~$j$ and $r_j=0$ otherwise. Thus the state $r=(0,\ldots,0)$ represents complete health. Denote by $\xR$ the set of possible states. Suppose that a dataset $\xD$ is available that contains the disease state for $N$ individuals. We view $\xD$ as $n$ independent and identically distributed (i.i.d.)\ samples from a population distribution $p$, and our goal is to compute an estimate of $p$.

If there is sufficient data, then for each $r\in\xR$ we can estimate $p(r)$ by the maximum likelihood estimator $\phat_{\text{ML}}(r)=N_r/N$, where $N_r$ is the number of individuals in $\xD$ whose state equals $r$. That is, $\phat_{\text{ML}}$ is simply the empirical distribution of the data. In practice, the number of data points $N$ is typically much less than the $2^k$ possible disease states. While for some states $r$ there will be enough data to adequately estimate $p(r)$ by $\phat_{\text{ML}}(r)$ as above, for most states there is insufficient data to form an estimate; indeed, we will erroneously conclude that many states have probability 0. (One strategy is to simply augment the data by adding an artificial data point for each missing $r$ value, but we seek a more principled approach.)

To further complicate the problem, it is often desirable to expand the state definition to include covariates associated with the diseases, such as age, body mass index, smoker versus non-smoker and so on. The presence of these additional features can dramatically increase the size of the state space $\xR$. Moreover, the estimation problem is more complicated, in that features can be real-valued, integer, or categorical. 

In the case where all features are discrete (integer or categorical), we estimate $p$ by computing a small, carefully chosen set of reliable statistics from the training data $\xD$ and using these statistics to develop a parametric model from which an approximate version of $p(r)$ can be computed. The key idea is to carefully select a set $\xC$ of important \emph{constraints} that are satisfied by $p$, and then to approximate $p$ by the ``simplest'' distribution that obeys the constraints in $\xC$; i.e., we impose no further implicit or explicit assumptions. Following standard practice~\cite{jaynes82}, we formalize the notion of ``simplest distribution'' as the distribution $p$ satisfying the given constraints that has the maximum \emph{entropy} value $H(p)$, where
\[
H(p)=-\sum_{r\in\xR}p(r)\log p(r).
\]
This maximum entropy (maxEnt) approach is well known in the mach\-ine-learn\-ing and linguistics communities---see \cite{Malouf02}---and can also be viewed as a principled approach to smoothing the empirical distribution. To deal with continuous features, we use kernel density estimation, conditioned on the value of the discrete features.

A desirable feature of the maxEnt framework is that it allows a ``pay as you go'' approach that controls the tradeoff between the resolution of the model and the cost of computing the model. The more data points and constraints, the greater the accuracy of the smoothed approximation. In the following sections, we provide the details of the approach.

\section{Basic MaxEnt for Discrete Features}

We first consider the simplest case with no covariates, so that each state $r$ comprises $k$ binary values. We then extend our methods to arbitrary discrete features and continuous features.

\subsection{Binary Features}

In the binary setting, the maxEnt problem can be stated as follows. Let $\xP$ denote the set of all possible probability distributions over $\xR$. For each subset $\xR_c\subseteq\xR$, denote by $f_c$ the indicator function of $\xR_c$, so that $f_c(r)=1$ if $r\in\xR_c$ and $f_c(r)=0$ otherwise. We number these subsets from 1 to $2^{|\xR|}$. Set $a_c=M_c/N$, where $M_c$ is the number of the $N$ data points $r\in\xD$ that lie in $\xR_c$ and thus satisfy $f_c(r)=1$. If, as in the previous section, we define $N_r$ as the number of persons in $\xD$ having disease state $r$, then we can also compute $a_c$ as
\begin{equation}\label{eq:defac}
a_c=\frac{1}{N}\sum_{r\in\xR}N_rf_c(r).
\end{equation}
The maxEnt problem is then
\begin{equation}\label{eq:maxent}
\begin{split}
&\maximize_{p\in\mathcal{P}}H(p)\text{ such that}\\
&\sum_{r\in\xR}f_c(r)p(r)=a_c\text{ for }c\in\xC,
\end{split}
\end{equation}
Where $\xC\subseteq [1..2^{|\xR|}]$ is the set of maxEnt constraints. For each $c\in C$, the corresponding maxEnt constraint requires that $p(\xR_c)=a_c$. That is, the probability of $\xR_c$ must match the empirical probability of $\xR_c$ in $\xD$, and we seek the simplest distribution that satisfies the constraints in $\xC$.

The most basic constraints in $\xC$ specify the marginal occurrence probability for each disease. That is, for each disease~$i$, the corresponding constraint function $f_c$ satisfies $f_c(r)=1$ if and only if $r_i=1$, so that $\xR_c=\{\,r\in\xR:r_i=1\,\}$. Equivalently, if $R$ denotes a random sample from $p$, then the constraint can be expressed as $\prob{R_i=1}=N_i/N$, where $N_i$ is the fraction of individuals in $\xD$ having disease~$i$. We include such constraints because of their
inherent importance and because we can obtain a relatively accurate estimate of $a_c$ from the training data; indeed, most diseases will appear in a non-negligible fraction of individuals in $\xD$, so data sparsity is not an issue.

Other constraints can be added to $\xC$, corresponding to a small number of ``important'' statistical features of $p$ as specified by the user. Typically, these important features specify statistical dependencies between the various diseases that should be respected in any approximation to $p$. For example, denoting by $N_{ij}$ the number of persons in $\xD$ having both diseases~$i$ and $j$, one might specify a constraint of the form
\[
\prob{R_i=1\wedge R_j=1}=a_{ij},
\]
where $a_{ij}=N_{ij}/N$ is close to 1, in order to capture a known, strong positive dependency between the diseases. Similarly, a constraint of the form
\[
\prob{R_i=1\wedge R_j=0}=a_{i\bar{\jmath}},
\]
can capture a strong negative dependency---persons with disease~$i$ tend not to have disease~$j$. As above, these constraints can be expressed in terms of indicator functions $f_c$. Note that the constraints, being based on a finite set of patient data, are approximate rather than exact; we return to this issue below. Also note that, in principle, the user can directly specify the $a_c$ constants based on expert knowledge about rule behavior. The user must then ensure, however, that the resulting set $\xC$ of constraints is consistent; see, e.g., Markl et al.~\cite{Markl2007} for a discussion of this issue. To avoid situations with insufficient data, to avoid overfitting, and to control computational costs, we typically include only bivariate probabilities and only for the most correlated feature pairs.

\subsection{Discrete Features}

The foregoing setup can be extended to arbitrary discrete features. The idea is to encode a given feature $s$ having $m$ possible values as a binary vector $v$ of length~$m$, where $v_i=1$ if and only if the feature takes on the $i$th possible value. The resulting set of marginal maxEnt constraints require that the marginal distribution of $s$ matches the empirical distribution in $\xD$. (Note that we add an additional constraint which requires that the sum of $v_i$'s equal 1 with probability~1.) If $m$ is very large, making such an encoding impractical, one alternative for an integer-valued feature is to replace the $m$ maxEnt constraints with the single constraint that the expected value of $s$ under $p$ matches the average value in $\xD$. Additional constraints corresponding to other features of the distribution of $s$, such as higher moments or tail probabilities, can also be added. A potential drawback to this approach is that such aggregate constraints can severely underspecify the distribution; see~\cite{YuDA09}. An alternative approach aggregates feature values into bins, but in practice the maxEnt fit can be extremely sensitive to the bin specification. One simple strategy is to approximate the distribution of $s$ by a continuous distribution and use, e.g., a kernel density approach as described later.

\subsection{Computing the maxEnt distribution}

The standard approach to computing the maxEnt distribution converts the problem \eqref{eq:maxent} into an equivalent formulation, based on duality theory for convex optimization problems. Specifically, assuming that $a_c$ is estimated from the training data as discussed previously, it can be shown---see, e.g., \cite{PietraPL97}---that the entropy-maximizing distribution must be of the form
\begin{equation}\label{eq:expmodel}
p(r)=p(r;\theta)=\frac{1}{Z(\theta)}\exp\Bigl\{\sum_{c\in\xC}\theta_c f_c(r)\Bigr\},
\end{equation}
where $\theta=\{\,\theta_c:c\in\xC\,\}$ and $Z(\theta)=\sum_{r\in\xR}\exp\bigl\{\sum_{c\in\xC}\theta_c f_c(r)\bigr\}$ is a normalizing constant which ensures that $\sum_{r\in\xR}p(r)=1$. Moreover, solving \eqref{eq:maxent} is equivalent to solving the problem 
\begin{equation}\label{eq:maxlike}
\maximize_{\theta_c:c\in\xC}\sum_{r\in\xR}N_r\sum_{c\in\xC}\theta_cf_c(r)-N\log Z(\theta),
\end{equation}
which can be viewed as computing a maximum likelihood estimate of $\theta$ based on the data in $\xD$. The optimization problem  \eqref{eq:maxlike} typically must be solved numerically; as discussed in \cite{Malouf02}, standard optimization algorithms such as L-BFGS-B~\cite{ZhuBLN97} tend to outperform techniques such as iterative scaling, even though the latter techniques were developed expressly for maxEnt problems.

\subsection{Avoiding Overfitting}

Since the data used to fit the population distribution $p$ is based on a finite sample, some care must be taken to avoid overfitting our model for $p$. Several approaches to this problem have been proposed in the literature. Perhaps the simplest approach is to take a Bayesian viewpoint and introduce a ``prior'' distribution on the parameters in $\theta$ and then use a maximum posterior estimate rather than a maximum likelihood estimate as above. In the case of a gaussian prior, this reduces to solving a modified version of \eqref{eq:maxlike}:
\begin{equation}\label{eq:maxlikegauss}
\maximize_{\theta_c:c\in\xC}\sum_{r\in\xR}N_r\sum_{c\in\xC}\theta_cf_c(r)-N\log Z(\theta) -\sum_c (\theta^2_c/ 2\sigma^2_c),
\end{equation}
where for each $c\in\xC$ the parameter $\sigma^2_c$ is the variance of the gaussian prior for $\theta_c$; see~\cite{ChenR00}. The $\sigma^2_c$ values can be selected using cross-validation. Goodman~\cite{Goodman04} proposes the use of an exponential prior and claims improved accuracy on certain NLP problems. An alternative approach~\cite{KazamaT05} replaces the equality constraints in the maxEnt problem \eqref{eq:maxent} by box-type inequality constraints. I.e., the constraints are now of the form
\[
\sum_{r\in\xR}f_c(r)p(r)\in[a_c-\eps_c,a_c+\eps_c],
\]
where the half-width $\eps_c$ is chosen to reflect the uncertainty in $a_c$ as an estimator of $p(\xR_c)$.

\subsection{Continuous Features}\label{sec:contf}

The discussion so far has centered around discrete-valued features, but continuous-valued features often appear as disease covariates. The challenges engendered by discrete features are similar to those associated with discrete features having a very large domain. As in the latter case, problems arise in approaches based either on specifying constraints for summary statistics or on binning of values. Yu et al.~\cite{YuDA09} propose an approach that, roughly speaking, uses binning but lets the number of bins go to infinity, yielding a continuous weight function $\theta$. This continuous functions is approximated via cubic splines, yielding a standard maxEnt problem but with a set of $K$ transformed features corresponding to each original continuous features. Besides blowing up the number of features, the derived features are not directly interpretable, which can hinder understanding of the maxEnt model.

An alternative, simple approach directly uses kernel density methods---see, e.g., \cite{Scott92}---to estimate the conditional distribution of the continuous features, given the values of the discrete features. This approach works well only when there are not too many continuous or discrete features. As discussed in Section~\ref{sec:factor} below, however, we typically break up the maxEnt model (either exactly or approximately) into independent submodels, and each submodel typically contains a small number of features. For convenience, write $r=(u,v)$, where $u$ and $v$ comprise the continuous and discrete feature values, and set $\xR_d=\{\,v:(u,v)\in\xR\,\}$. The idea is to compute, for each $v\in\xR_d$, a kernel density estimator of the joint conditional distribution $p_c(\darg\mid v)$ of the continuous features, given that the subvector of discrete features equals $v$. For $r=(u,v)\in\xR$, denote by $l$ the number of continuous features and set $\pi_c(r)=u$ and $\pi_d(r)=v$, so that $\pi_c$ and $\pi_d$ are the projections onto the continuous and discrete components of $r$. Then set 
\[
p_c(u\mid v)=\frac{1}{|\xD_v|}\sum_{r\in\xD_v}|H|^{-1/2}K\Bigl(H^{-1/2}\bigl(u-\pi_c(r)\bigr)\Bigr),
\]
for $v\in\xR_d$, where $\xD_v=\{\,r\in\xD:\pi_d(r)=v\,\}$. Here $K$ is a \emph{kernel} function that smooths out the influence of each data point in $\xD_v$, and $H$ is an $l\times l$ symmetric and positive definite \emph{bandwidth matrix} that controls how fast the influence of a data point decreases with distance from the point. A typical choice of kernel function is standard multivariate gaussian: $K(x)=(2\pi)^{-l/2}e^{-\Vert x\Vert^2/2}$. Generally, the choice of bandwidth is more important than the particular form of kernel. Many options are available---one common choice is to take $H$ to be a diagonal matrix and then determine each of the diagonal elements using the data driven ``plug-in'' bandwidth formula of Sheather and Jones~\cite{SheatherJ91}, which was subsequently improved in \cite{LiaoWL10}. See \cite{WandJ94} for a treatment of plug-in formulas for general bandwidth matrices.

It may be the case that $|\xD_v|$ is small for some $v\in\xR_d$, so that there are not enough data points on which to fit the conditional distribution. In this case, we can simply fit a distribution to the pooled data in $\xD=\bigcup_v\xD_v$. A more sophisticated approach might greedily drop one feature at a time out of $v$ until a sufficient number of training points are obtained.

\section{Performance Optimizations}

Fitting maxEnt models can be expensive when the number of features $k$, constraints $|\xC|$, or number of data points $N$ is large. In this section we discuss exact and approximate methods for speeding up maxEnt distribution fitting.

\subsection{Sampling Approach for Large $N$}

For large $N$, Schofield~\cite{Schofield04} provides a sampling-based method for quick approximate calculation of $Z(\theta)$ and $\{\,a_c:c\in\xC\,\}$ during optimization iterations.  These methods can be used to handle continuous features as well, but only when the corresponding constraints are on the values of summary statistics. As mentioned previously, such constraints generally under-specify the distribution.

\subsection{Factorization for Large $|\xC|$ and $k$}\label{sec:factor}

An exact method for accelerating maxEnt computations decomposes the set of $k$ features into $m>1$ partitions. Specifically, write $i\leftrightarrow j$ if and only if there exists at least one maxEnt constraint that involves both feature~$i$ and feature~$j$, and denote by $\leftrightsquigarrow$ the transitive closure of $\leftrightarrow$. Then features~$i$ and $j$ belong to the same partitions if and only if $i\leftrightsquigarrow j$. We correspondingly partition the state as $r=(r^{(1)},\ldots,r^{(m)})$. Recall that, in the absence of constraints, features are independently distributed under the maxEnt distribution. In a similar manner, the features within a given component $r^{(i)}$ will be statistically independent, but variables in different partitions will be independent; see, e.g., \cite[App.~B]{Markl2007}. Therefore, we can fit a maxEnt model $p^{(i)}$ to the data for each component $r^{(i)}$ and compute $p$ as $p(r)=p^{(1)}(r^{(1)})\times\cdots\times p^{(m)}(r^{(m)})$.

As mentioned previously, if we include multi-feature constraints for only the most important statistical dependencies, then the maxEnt model will decompose into many small submodels, each involving only a few features. This decomposition can greatly reduce the computational burden, since each state space will be relatively small. Indeed, many partitions might contain exactly one feature, in which case the corresponding distribution submodel can trivially be computed analytically. If a partition is too large to be computed in reasonable time, then approximate methods can be used. One popular method is the ``piecewise likelihood'' method in \cite{Sutton2007}. Roughly speaking, the sets of features referred to in the various partition-based constraints are initially treated as if they were independent (even though the sets are not even disjoint), and then dependence is re-introduced via a global normalization constant.

\section{Choosing maxEnt Constraints}

As can be seen from the previous section, a judicious choice of maxEnt constraints can result in good computational efficiency via factorization of the maxEnt model. The key challenge is to identify the most important statistical dependencies between features. Typically we focus on the most important bivariate dependencies, since these lead to constraints for which the $a_c$ constant can be reliably estimated. We need a notion of statistical dependency that will encompass integer, categorical, or continuous data. An appealing candidate is the information-theoretic  ``L-measure'' defined by Lu~\cite{Lu11} as follows. For a pair or random variables $(X,Y)$ with joint distribution $\gamma(x,y)$, let $I(X;Y)$ be the mutual information between $X$ and $Y$, defined as
\[
I(X;Y)=\sum_{x,y}\gamma(x,y)\log\frac{\gamma(x,y)}{\gamma(x)\gamma(y)}.
\]
Denote by $A_{X,Y}$ the set of all pairs of random variables $(U,V)$ having the same marginal distributions as $X$ and $Y$, and set $W(X;Y)=\sup_{(U,V)\in A_{X,Y}} I(U;V)$. We have
\[
W(X;Y)=
\begin{cases}
\min\bigl(H(X),H(Y)\bigr)&\text{$X$, $Y$ discrete};\\
H(Y)&\text{$X$ continuous, $Y$ discrete};\\
H(X)&\text{$X$ discrete, $Y$ continuous};\\
+\infty&\text{$X$, $Y$ continuous},
\end{cases}
\]
where $H$ denotes entropy as before. Then L-measure is defined as
\[
L(X,Y)=\Biggl[1-\exp\Bigl\{\frac{-2I(X;Y)}{1-I(X;Y)/W(X;Y)}\Bigr\}\Biggr]^{1/2}.
\]
In the fully continuous case, L-measure reduces to the Linfoot measure~\cite{Linfoot57}. L-measure has a number of desirable properties. By definition, it applies to integer, categorical, and continuous features. It is symmetric in $X$ and $Y$ and takes values in $[0,1]$, where a value of 0 implies independence between $X$ and $Y$, and a value of 1 implies a strict functional relationship (not necessarily linear) between the two variables. L-measure is invariant under continuous, strictly increasing transformations and, in the special case where $(X,Y)$ is bivariate gaussian, it reduces to the absolute value of the correlation coefficient.
 
 To compute the needed entropy and mutual information values needed for L-Distance, available techniques include the ``shrinkage'' estimator of Hausser and Strimmer~\cite{HausserS09} for the discrete-discrete case, the nearest neighbor estimator of Kraskov et al.~\cite{KraskovSG03} for the continuous-continuous case, and a modified version of this estimator due to Ross~\cite{Ross14} for the mixed discrete-continuous case. For discrete features having many distinct values and few duplicated values in the data set, the Ross estimator fails; in this case, the discrete feature can be treated as continuous and the estimator of Kraskov et al.\ can be applied. 
 
The presence of a discrete feature having many distinct values also complicates the use of L-measure. Specifically, if a discrete feature~$X$ has very many distinct values, it's mutual information with another discrete feature~$Y$ may be close to 1 even if features~$X$ and $Y$ are in fact independent. To see why, first observe that if we treat a continuous feature~$X$ as discrete, then its mutual information with any other discrete feature $Y$ will equal 1. The reason is that, given a set of $(x,y)$ pairs of feature values, the $x$'s are all distinct, so that a value $x$ uniquely determines its corresponding $y$. A discrete feature having many distinct values behaves in roughly the same way, leading to erroneously high values of mutual information.

A solution to this problem can be found by normalizing the mutual information values ``with respect to chance'', as in Nguyen et al.~\cite{NguyenEB09}. Given a set $\xS$ of i.i.d.\ samples $(x,y)$ from $(X,Y)$,\footnote{We assume throughout that every possible marginal value of $X$ and of $Y$ is represented in $\xS$.} we first compute $\mu(X;Y)$, the expected value of $I(X;Y)$ under a random permutation of the $y$ values in the set of pairs. When $X$ and $Y$ are discrete, this random permutation model can also be viewed as randomly and uniformly selecting a contingency table $M$ from the set $\xM$ of all contingency tables whose marginal row and column totals are the same as that of the table $M_0$ that corresponds to the original data in $\xS$. Defining marginal totals $n_{i\cdot}=|\{\,(x,y)\in\xS:x=x_i\,\}|$ and $n_{\cdot j}=|\{\,(x,y)\in\xS:y=y_j\,\}|$ and setting $N=|\xS|$, Nguyen et al. show that
\begin{equation}\label{eq:muxy}
\mu(X;Y)=\sum_{i,j}\sum_{n_{ij}}\frac{n_{ij}}{N}\log\Bigl(\frac{n_{ij}N}{n_{i\cdot}n_{\cdot j}}\Bigr)P(n_{ij}),
\end{equation}
where the outer sum runs over all values $x_i$ and $y_j$ of $X$ and $Y$, the inner sum runs from $n_{ij}=\max(0,n_{i\cdot}+n_{\cdot j}-N)$ to $n_{ij}=\min(n_{i\cdot},n_{\cdot j})$, and
\[
P(n_{ij})=\Bigl(\frac{N-n_{\cdot j}}{n_{i\cdot}-n_{ij}}\Bigr)\Bigl(\frac{n_{\cdot j}}{n_{ij}}\Bigr)\bigg/\Bigl(\frac{n}{n_{i\cdot}}\Bigr).
\]
When at least one of the features is continuous, we can estimate $\mu(X;Y)$ via Monte Carlo. We then compute the normalized mutual information as $\yI(X;Y)=I(X;Y)-\mu(X;Y)$, and define the normalized L-measure as
\[
\yL(X,Y)=\Biggl[1-\exp\Bigl\{\frac{-2\yI(X;Y)}{1-\yI(X;Y)/\yW(X;Y)}\Bigr\}\Biggr]^{1/2},
\]
where $\yW(X;Y)=\sup_{(U,V)\in A_{X,Y}} \yI(U;V)=W(X;Y)-\mu(X;Y)$.

For a pair of discrete features $(X,Y)$ that is determined to be highly dependent according to normalized L-measure, we need to identify the most important $(x_i,y_j)$ value pairs to use in maxEnt constraints. This can be achieved by ranking the pairs with respect to their absolute contribution to the empirical mutual information, i.e., rank according to
\[
\delta(x_i,y_j)=\Bigl\vert \frac{n_{ij}}{n}\log\Bigl(\frac{n_{ij}n}{n_{i\cdot}n_{\cdot j}}\Bigr)\Bigr\vert,
\] 
where $n_{i\cdot}$, $n_{\cdot j}$ and $n$ are as \eqref{eq:muxy} and $n_{ij}=|\{\,(x,y)\in\xS:x=x_i\text{ and }y=y_j\,\}|$. For each $(x_i,y_j)$ pair selected, the corresponding maxEnt constraint equates the joint probability that $X=x_i$ and $Y=y_j$ to the empirical joint frequency in the data. The number of such $(x_i,y_j)$ pairs is chosen so as to balance model accuracy with computational cost. 

\section{Fitting the Distribution}

Putting the previous steps together, we fit a smoothed distribution to the data set $\xD$ as follows.
\begin{enumerate}
\item Compute normalized L-measures between all feature pairs, and select the $k$ most dependent pairs, where $k$ is chosen to trade off accuracy and computational cost. The value of $k$ may also be chosen indirectly using a cutoff value for L-measure, where the cutoff is chosen to balance accuracy and cost. For each selected pair of discrete features, select specific pairs of feature values to use in maxEnt constraints, as described at the end of the previous section.
\item Partition the features to create one or more submodels. Features are partitioned as described in Section~\ref{sec:factor}, except that we extend the partitioning scheme to include continuous features by now writing $i\leftrightarrow j$ if and only if $(i,j)$ is one of the $k$ most-dependent pairs selected in Step~1. Steps 3--5 are now performed for each submodel~$i$, where $i=1,2,\ldots,m$. For submodel~$i$, write $r^{(i)}=(u^{(i)},v^{(i)})$, where $u^{(i)}$ an $v^{(i)}$ are the sets of continuous and discrete feature values for state $r^{(i)}$. Denote by $\xR_d^{(i)}$ the state space of $v^{(i)}$.
\item Encode all non-binary discrete features as binary vectors (perhaps recategorizing features with many distinct values as effectively continuous), and rewrite the maxEnt constraints accordingly.
\item Solve the optimization problem \eqref{eq:maxlike} to fit the $\theta_c$ parameters and hence compute the joint distribution $p_d^{(i)}(v^{(i)})$ of the binary features.
\item For each $v^{(i)}\in\xR_d^{(i)}$, compute a conditional kernel density estimator $p_c^{(i)}(u^{(i)}\mid v^{(i)})$ of the joint distribution of the continuous features. If there are too few data points to fit $p_d^{(i)}$ reliably, then fit a density estimator to pooled data as discussed in Section~\ref{sec:contf}.
\item For $r^{(i)}=(u^{(i)},v^{(i)})$, define the submodel~$i$ as $p^{(i)}(r^{(i)})=p_d^{(i)}(v^{(i)})p_c^{(i)}(u^{(i)}\mid v^{(i)})$.
\item Define the overall model as $p(r)=p^{(1)}(r^{(1)})\times\cdots\times p^{(m)}(r^{(m)})$.
\end{enumerate}



\bibliographystyle{abbrv}
\bibliography{maxent}

\end{document}  